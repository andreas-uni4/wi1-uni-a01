\documentclass[12pt,utf8]{scrartcl}
\usepackage[ngerman]{babel}
\usepackage{hyperref}
\hypersetup{
	colorlinks=true,	
	linkcolor=blue,     
	citecolor=blue,     
	filecolor=blue,     
	urlcolor=blue     	
}
\usepackage{etoolbox}
\apptocmd{\UrlBreaks}{\do\-\do\%\do\.}{}{}
\usepackage[ngerman]{varioref}
\usepackage{amsmath,amssymb,latexsym,amsfonts,amsthm,amsbsy,qtree}
\usepackage{url}
\usepackage[printonlyused]{acronym}
\usepackage[utf8]{inputenc} 
\usepackage{graphicx}
\usepackage{float}
\usepackage{fancyhdr}
\usepackage{booktabs}
\usepackage{enumitem}
\usepackage[justification=centering]{caption}
\usepackage{natbib}
\bibliographystyle{plain}
\usepackage{pdfpages}


\newcommand{\teilnehmerI}{Tom Dombeck}
\newcommand{\mattI}{4510671} 
\newcommand{\mailI}{todo@uni-bremen.de}
\newcommand{\teilnehmerII}{Andreas Schwarz}
\newcommand{\mattII}{4250572}
\newcommand{\mailII}{andreas4@uni-bremen.de}
\newcommand{\teilnehmerIII}{Lasse Warnke}
\newcommand{\mattIII}{4515161}
\newcommand{\mailIII}{lwarnke@uni-bremen.de}
\newcommand{\thisgroup}{A01}
\newcommand{\abgabedatum}{19.11.2018}
\newcommand{\nummer}{1}
\newcommand{\thema}{Hype-Cycle-Themen}
\newcommand{\thistutor}{Tim Haß}
\newcommand{\thissemester}{WiSe 2018/19}
\newcommand{\thiscourse}{Wirtschaftsinformatik 1}
\newcommand{\thisshortcourse}{WI1}


\pagestyle{fancy}
\fancyhead{} 												
	\fancyhead[LO,RE]{\thissemester \\ \thisshortcourse} 
	\fancyhead[RO,LE]{TutorIn: \thistutor \\ Gruppe: \thisgroup }
	\fancyfoot{} 											
	\cfoot{\thepage} 										
	\setlength{\headsep}{2cm} 								


\begin{document}
\begin{titlepage}
	\vspace*{\baselineskip}		
	\centering					
	\LARGE							
	\thiscourse \\ 					
	\vspace{1cm}					
	{\Huge 							
	\textbf{Abgabe \nummer: \thema}} \\ 
	\vspace{1.5cm} 					
	TutorIn: \thistutor \\ 		
	\abgabedatum \\ 				
	\vfill 							
	Gruppe: \thisgroup \\ 			
	\vspace{.5cm} 					
	\large 							
	\begin{tabular}{c|c|c} 		
	\teilnehmerI	& \teilnehmerII & \teilnehmerIII \\ 
	\mattI	& \mattII &  \mattIII\\ 
	\mailI	& \mailII & \mailIII \\ 
	\end{tabular} 
\end{titlepage}


\thispagestyle{empty}
\tableofcontents
\newpage
\begin{flushleft}
\setcounter{page}{1}


\section{\label{sec:thema}Aufgabe 1.1: Augmented Reality}
\subsection{\label{sub:thema}a.) Beschreibung von Themegebiet}

...Beschreibung...

\subsection{\label{sub2:thema}b.) eigene Definition}

...Definition...

\subsection{\label{sub3:thema}c.) Ausprägungen und Abwandlungen}

...Beschreibung...

\subsection{\label{sub4:thema}d.) mögliche zukünftige Entwicklung}

...Text...


\section{\label{sec:einfuehrung}Aufgabe 1.2: Unterschiede und Gemeinsamkeiten beim Verständnis von Wirtschaftsinformatik}
\subsection{\label{sub:einfuehrung}a.) Kriterien für einen Vergleich}

Forschungsgegenstand

Forschungsergebnisse

Praxisorientierung

Forschungsmethoden

Bedeutung

Zielgruppe
\newpage
\subsection{\label{sub2:einfuehrung}b.) Vergleich von den Zeitschriften HMD und BISE}

\begin{tabular}{|p{4cm}|p{5.5cm}|p{5.5cm}|}
\hline
& HMD & BISE \\
\hline
Forschungsgegenstand & Entwicklung und verbesserung von Geschäftsprozessen und IS & Entwicklung und verbesserung von Geschäftsprozessen und IS \\
\hline
Forschungsergebnisse & praxisrelevante Ergebnisse & praxisrelevante Ergebnisse \\
\hline
Praxisorientierung & sehr wichtig & sehr wichtig \\
\hline
Forschungsmethoden & weniger wichtig & weniger wichtig \\
\hline
Qualität/ Wichtigkeit & international eher wenig bedeutend & international wenier bedeutend \\
\hline
Zielgruppe & Fach- und Führungskräfte, Studierende und Lehrende (in Deutschland) & Fach- und Führungskräfte, Studierende und Lehrende (hauptsächlich in Deutschland und Europa) \\
\hline
\end{tabular}
\newline
\newline
\newline

Bei den beiden zu vergleichenden Zeitschriften \emph{HMD Praxis der Wirtschaftsinformatik} und \emph{BISE / WIRTSCHAFTSINFORMATIK} handelt es sich um Fachzeitschrifften aus dem Gebiet der Wirtschaftsinformatik. Bei beiden Zeitschriften handelt es sich um deutsche Veröffentlichungen. Wobei die \emph{BISE} seit 2015 nur noch in Englischer Sprache herausgebracht wird\cite{BISE}.

Sowohl die \emph{HMD} als auch die \emph{BISE} beschäftigen sich mit Herausforderungen und deren Lösungsideen sowie deren Umsetzungsmöglichkeiten\cite{Meier2017}\cite{Stein2017}\cite{Ebert2017}. Wie der Name der \emph{HMD Praxis der Wirtschaftsinformatik} schon sagt sind die Ergebnisse aus Veröffentlichungen sehr praxisrelevant. Die trifft ebenfalls auf die \emph{BISE} zu. Artikel aus beiden Fachzeitschriften beziehen sich auf praktische Themen deren Praxistauglichkeit und teilweise auch wie diese wirklich umgesetzt werden können\cite{Meier2017}\cite{Stein2017}\cite{Ebert2017}\cite{Kakar2017}. Ein Beispiel für die Praxisorientiertheit der Zeitschriften ist der Artikel \emph{Spielerisch lockt der Einzelhandel den Kunden – Einfluss von Belohnungen auf die Kanalwahl}\cite{Stein2017} aus der \emph{HMD} von 2017. Er bezieht sich auf den Einsatz von Gamification im Einzelhandel und dessen Nutzen. 

Wenn es um die Bedeutung geht ist die \emph{HMD} internatilonal gesehen eine eher unbedeutende Zeitschrift (VHB-JOURQUAL-Ranking von D)\cite{VHB}. Allerdings wird sie auch hauptsächlich in Deutsch veröffentlicht. Daher ist sie auch hauptsächlich im seutschsprachigen Raum relevant. 
Die \emph{BISE} ist im Vergleich zur \emph{HMD} international schon deutlich relevanter (VHB-JOURQUAL-Ranking von B)\cite{VHB}. Allerdings ist sie trotz Veröffentlichung in englischer Sprache, aufgrund der starken Neigung zur deutschen Wirtschaftsinformatik, vor allem in Europa relevant. 

Die Zielgruppe bei beiden Zeitschrifften wieder ziemlich ähnlich. Beide Zeitschrifften sind an Fach- und Führungskräfte sowie an Studierende und Lehrende gerichtet. Nur, dass die \emph{HMD} sich aufgrund der Sprache eher an den deutschsprachigen Raum und die \emph{BISE} ebenfalls an Englischsprachige richtet.
\newline

{\Large Fazit}

Inhaltlich sowie methodisch ähneln sich beide Zeitschrifften sehr stark. Dies dürfte vorallem daran liegen, dass die \emph{HMD} und die \emph{BISE} sowohl aus dem selben Fachgebiet (Wirtschaftsinformatik) als auch aus der selben Ragion (Deutschland) kommen.
\subsection{\label{sub3:einfuehrung}c.) Vergleich von den Zeitschriften MISQE und ISR}

\begin{tabular}{|p{4cm}|p{5.5cm}|p{5.5cm}|}
\hline
& MISQE & ISR \\
\hline
Forschungsgegenstand & Wichtiges und Nützliches für die Praxis der Information Systems & Theorie und Hintergründe der Information Systems \\
\hline
Forschungsergebnisse & teils sehr Praxisnah & kaum Praxisbezug \\
\hline
Praxisorientierung & wichtig & teils vorhanden, teils nicht \\
\hline
Forschungsmethoden & wichtig & wichtig \\
\hline
Qualität/ Wichtigkeit & international eine wichtige und geschätzte Zeitschrift in dieser Fachrichtung & international eine der am meisten bedeutenden Zeitschriften dieser Fachrichtung \\
\hline
Zielgruppe & Manager und Investoren und andere Führungskräfte sowie Forschende &  \\
\hline
\end{tabular}
\newline
\newline
\newline

Das Journal \emph{Information Systems Research} gehört zu den führenden Zeitschriften in der Fachrichtung der \emph{Information Systems} und wird auf der ganzen Welt anerkannt. Das Journal \emph{MIS Quarterly Executive} ist eine Schwester Veröffentlichung von \emph{MISQ}, ebenfalls eine der größten Fachzeitschriften im Gebiet der \emph{Information Systems}, ist selbst allerdings nicht ganz so anerkannt und gefragt, genießt aber dennoch einen relativ hohen Stellenwert.

Die beiden Journals unterscheiden sich grundlegend bezüglich ihrer Forschungsgegenstände. Während \emph{MISQE} darauf ausgelegt ist Wichtiges und Nützliches für die Praxis der Information Systems zu behandeln und erforschen, behandelt die \emph{ISR} eher die Theorie und Hintergründe der Information Systems. Daraus folgt auch unweigerlich, dass die Ergebnisse, die \emph{MISQE} liefert, deutlich Praxisrelevanter sind, als Ergebnisse von \emph{ISR}. Die Artikel aus \emph{MISQE} gehen auf Probleme in der Wirtschaft ein und geben Gründe und Empfehlungen für diese an bzw. versuchen die Zukunftsfähigkeit zu beurteilen. Die Artikel aus der \emph{ISR} hingegen gehen vorwiegend auf die Hintergründe ein. 

Wie bereits in der Einleitung erwähnt sind beide Journals recht angesehen im Gebiet der Information Systems. Dies verdeutlicht auch das VHB-Journal-Ranking der Wirtschaftsinformatik. Hier belegt \emph{ISR}, eine "herausragende, weltweit führende wissenschaftliche Zeitschrift"\ den ersten Rang mit einer Spitzennote von A+, während \emph{MISQE} immerhin noch ein B bekommt, also "nur"\ eine "wichtige und angesehene wissenschaftliche Zeitschrift"\ ist\cite{VHB}.

Auch in der Zielgruppe unterscheiden sich beide Journals leicht. Während sich \emph{MISQE} aufgrund der starken Praxisorientierung eher an Manager, Investoren und andere Führungskräfte wendet und nur sekundär an die Forschung, ist es bei \emph{ISR} genau anders herum. Aufgrund der eher theoretischen Ergebnisse ist dieses Journal eher in der Forschung relevant. 
\newline

{\Large Fazit}

Zwischen \emph{MISQE} und \emph{ISR} gibt es doch einige bedeutende Unterschiede. Während \emph{MISQE} eher Praxisorientiert ist, konzentriert sich \emph{ISR} auf die Theorie. Allerdings sind beide Journals anerkannt und weltweit verbreitet. 



\subsection{\label{sub4:einfuehrung}d.) Vergleich von den Veröffentlichungen aus Deutschland (HMD und BISE) mit denen aus dem internationalen bzw. anglo-amerikanischen Bereich (MISQE und ISR).}

\begin{tabular}{|p{4cm}|p{5.5cm}|p{5.5cm}|}
\hline
& HMD und BISE & MISQE und ISR \\
\hline
Forschungsgegenstand & Entwicklung und verbesserung von Geschäftsprozessen und IS &  \\
\hline
Forschungsergebnisse & praxisrelevante Ergebnisse &  \\
\hline
Praxisorientierung & sehr wichtig &  \\
\hline
Forschungsmethoden & weniger wichtig &  \\
\hline
Qualität/ Wichtigkeit & international eher wenig bedeutend & international zwei der am meisten bedeutenden Zeitschriften dieser Fachrichtung \\
\hline
Zielgruppe & Manager und Investoren und andere Führungskräfte sowie Forschende &  \\
\hline
\end{tabular}
\newline
\newline
\newline

Test

\section{\label{sec:bonus}Bonusaufgabe}

Als Vorlage diente mir \cite{online1}  blabla. Geändert habe ich blabla. Logo selbst gemacht.
Icons von blabla\cite{online2}. Auf der nächsten Seite ist dann das ganze Stellenschreiben:

\newpage
\begin{figure}[htbp]
	\begin{minipage}[t]{4cm}
		\vspace{0pt}
		\centering
		\includegraphics{images/Logo}
		\label{fig:Logo}
	\end{minipage}
	\hfill
	\begin{minipage}[t]{4cm}
		\vspace{0pt}
		\scriptsize
		\textbf{DSW \& Partner}\\
		Mr. Mustermann\\
		Projektmanager\\
		Spaßbremen\\
		12345 Spaßstadt\\
		Tel. +00 1234 5678 \\
		E-Mail: bewerbungen@dsw.de
	\end{minipage}
\end{figure}
			
\large
\begin{center}Junior IT Consulant (m/w/divers) \\Business Intelligence\end{center}

\scriptsize
Als Berater im Bereich Business Intelligence erwartet Dich ein internationales Team aus Consultants und Softwareentwicklern aus ganz Europa. In Produktschulungen unserer DSW-Academy und in Weiterbildungsseminaren entwickelst Du Deine technischen und sozialen Kompetenzen weiter. Ob Patenprogramm, flexible Arbeitszeiten oder Home-Office – wir bieten Dir das Arbeitsumfeld, das Du für Deine persönliche Entwicklung benötigst.


\textbf{\\Wir haben spannende Aufgaben für Dich:}
\begin{center}
	\begin{itemize}
		\item Als Consultant für unser Business-Intelligence Softwareprodukt unterstützt Du unsere Kunden remote und vor Ort bei der Einführung unseres Kennzahlen- und ad-hoc-Analyse-Systems im IT-Service Management Kontext
		\item Dein Aufgabenfeld ist breit gefächert. Es reicht von der Analyse und Konzeption über die Umsetzung bis zum Training und der Produkteinführung. Dabei stehen einerseits die technische Beratung sowie andererseits die Umsetzung im Fokus
		\item Du arbeitest in Teilprojekten meist in größeren Implementierungsteams. Damit das Projekt erfolgreich umgesetzt werden kann ist auch der regelmäßige Austausch mit unserem Entwicklungsteam sehr wichtig
	\end{itemize}
\end{center}

\textbf{Das bringst Du mit:}
\begin{center}
	\begin{itemize}
		\item Du bringst ein abgeschlossenes Studium der (Wirtschafts-) Informatik oder vergleichbare Berufsausbildung mit
		\item Du hast bereits erste Berufserfahrung mit Business-Intelligence Systemen oder Dashboard-Systemen
		\item Du bringst Erfahrungen mit relationalen Datenbanken und SQL mit
		\item Du hast Freude daran, zusammen mit unseren Kunden individuelle Lösungen zu entwickeln und diese anschließend zu implementieren
		\item Du hast Spaß an der Zusammenarbeit im Projektteam und bist reisebereit
	\end{itemize}
\end{center}

\begin{center}
	\textbf{Deine Vorteile bei uns:}
\end{center}

\begin{center}
	\includegraphics{images/auto}\label{fig:auto}\hspace{1,5cm}
	\includegraphics{images/home}\label{fig:home}\hspace{1,5cm}	
	\includegraphics{images/handshake}\label{fig:handshake}\hspace{1,5cm}		
	\includegraphics{images/statistics}\label{fig:statistics}  
\end{center}
\hspace{0,3cm}\textbf{Eigener Firmenwagen,\hspace{0,7cm} Flexibel sein durch \hspace{0,9cm}Partnerschaftliches \hspace{1,1cm}Gute Karriere- \& \\ \hspace{0,4cm}auch Privat nutzbar\hspace{1,5cm}Home Office\hspace{2,1cm}Verhältniss \hspace{2cm}Aufstiegschancen} 			

\vspace{1cm}
\textbf{Wir haben dein Interesse geweckt?}\\
Dann schreib deine Bewerbung direkt per Email an mich oder bewirb Dich online über unser Karriere-Portal: www.dombeck-schwarz-warnke.de/DSW/stellenangebote/ – Die Stelle findest Du unter der Kategorie „Berufseinsteiger“.

\normalsize
\newpage
\addcontentsline{toc}{section}{Literaturverzeichniss}
\bibliography{Literaturdatenbank}


\end{flushleft}
\end{document}
