\documentclass[12pt,utf8]{scrartcl}
\usepackage[ngerman]{babel}
\usepackage{hyperref}
\hypersetup{
	colorlinks=true,	
	linkcolor=blue,     
	citecolor=blue,     
	filecolor=blue,     
	urlcolor=blue     	
}
\usepackage{etoolbox}
\apptocmd{\UrlBreaks}{\do\-\do\%\do\.}{}{}
\usepackage[ngerman]{varioref}
\usepackage{amsmath,amssymb,latexsym,amsfonts,amsthm,amsbsy,qtree}
\usepackage{url}
\usepackage[printonlyused]{acronym}
\usepackage[utf8]{inputenc} 
\usepackage{graphicx}
\usepackage{float}
\usepackage{fancyhdr}
\usepackage{booktabs}
\usepackage{enumitem}
\usepackage[justification=centering]{caption}
\usepackage{natbib}
\bibliographystyle{plain}
\usepackage{pdfpages}


\newcommand{\teilnehmerI}{Tom Dombeck}
\newcommand{\mattI}{4510671} 
\newcommand{\mailI}{todo@uni-bremen.de}
\newcommand{\teilnehmerII}{Andreas Schwarz}
\newcommand{\mattII}{4250572}
\newcommand{\mailII}{andreas4@uni-bremen.de}
\newcommand{\teilnehmerIII}{Lasse Warnke}
\newcommand{\mattIII}{4515161}
\newcommand{\mailIII}{lwarnke@uni-bremen.de}
\newcommand{\thisgroup}{A01}
\newcommand{\abgabedatum}{19.11.2018}
\newcommand{\nummer}{1}
\newcommand{\thema}{Hype-Cycle-Themen}
\newcommand{\thistutor}{Tim Haß}
\newcommand{\thissemester}{WiSe 2018/19}
\newcommand{\thiscourse}{Wirtschaftsinformatik 1}
\newcommand{\thisshortcourse}{WI1}


\pagestyle{fancy}
\fancyhead{} 												
	\fancyhead[LO,RE]{\thissemester \\ \thisshortcourse} 
	\fancyhead[RO,LE]{TutorIn: \thistutor \\ Gruppe: \thisgroup }
	\fancyfoot{} 											
	\cfoot{\thepage} 										
	\setlength{\headsep}{2cm} 								


\begin{document}
\begin{titlepage}
	\vspace*{\baselineskip}		
	\centering					
	\LARGE							
	\thiscourse \\ 					
	\vspace{1cm}					
	{\Huge 							
	\textbf{Abgabe \nummer: \thema}} \\ 
	\vspace{1.5cm} 					
	TutorIn: \thistutor \\ 		
	\abgabedatum \\ 				
	\vfill 							
	Gruppe: \thisgroup \\ 			
	\vspace{.5cm} 					
	\large 							
	\begin{tabular}{c|c|c} 		
	\teilnehmerI	& \teilnehmerII & \teilnehmerIII \\ 
	\mattI	& \mattII &  \mattIII\\ 
	\mailI	& \mailII & \mailIII \\ 
	\end{tabular} 
\end{titlepage}


\thispagestyle{empty}
\tableofcontents
\newpage
\begin{flushleft}
\setcounter{page}{1}


\section{\label{sec:thema}Aufgabe 1.1: Augmented Reality}
\subsection{\label{sub:thema}a.) Beschreibung von Themegebiet}

...Beschreibung...

\subsection{\label{sub2:thema}b.) eigene Definition}

...Definition...

\subsection{\label{sub3:thema}c.) Ausprägungen und Abwandlungen}

...Beschreibung...

\subsection{\label{sub4:thema}d.) mögliche zukünftige Entwicklung}

...Text...


\section{\label{sec:einfuehrung}Aufgabe 1.2: Unterschiede und Gemeinsamkeiten beim Verständnis von Wirtschaftsinformatik}
\subsection{\label{sub:einfuehrung}a.) Kriterien für einen Vergleich}

...sechs Kriterien....
\newpage
\subsection{\label{sub2:einfuehrung}b.) Vergleich von den Zeitschriften HMD und BISE}

\begin{tabular}{|p{4cm}|p{5.5cm}|p{5.5cm}|}
\hline
& HMD & BISE \\
\hline
Forschungsgegenstand & Entwicklung und verbesserung von Geschäftsprozessen und IS & Entwicklung und verbesserung von Geschäftsprozessen und IS \\
\hline
Forschungsergebnisse & praxisrelevante Ergebnisse & praxisrelevante Ergebnisse \\
\hline
Praxisorientierung & sehr wichtig & sehr wichtig \\
\hline
Forschungsmethoden & weniger wichtig & weniger wichtig \\
\hline
Qualität/ Wichtigkeit & international eher wenig bedeutend & international eher wenig bedeutend (allerdings bedeutender als HMD) \\
\hline
Zielgruppe & Fach- und Führungskräfte, Studierende und Lehrende (in Deutschland) & Fach- und Führungskräfte, Studierende und Lehrende (hauptsächlich in Deutschland und Europa) \\
\hline
\end{tabular}
\newline
\newline
\newline

Bei den beiden zu vergleichenden Zeitschriften \emph{HMD Praxis der Wirtschaftsinformatik} und \emph{BISE / WIRTSCHAFTSINFORMATIK} handelt es sich um Fachzeitschrifften aus dem Gebiet der Wirtschaftsinformatik. Bei beiden Zeitschriften handelt es sich um deutsche Veröffentlichungen. Wobei die \emph{BISE} seit 2015 nur noch in Englischer Sprache herausgebracht wird.  


\subsection{\label{sub3:einfuehrung}c.) Vergleich von den Zeitschriften MISQE und ISR}

...Vergleich...

\subsection{\label{sub4:einfuehrung}d.) Vergleich von den Veröffentlichungen aus Deutschland (HMD und BISE) mit denen aus dem internationalen bzw. anglo-amerikanischen Bereich (MISQE und ISR).}

...Vergleich...

\section{\label{sec:bonus}Bonusaufgabe}

Als Vorlage diente mir \cite{online1}  blabla. Geändert habe ich blabla. Logo selbst gemacht.
Icons von blabla\cite{online2}. Auf der nächsten Seite ist dann das ganze Stellenschreiben:

\newpage
\begin{figure}[htbp]
	\begin{minipage}[t]{4cm}
		\vspace{0pt}
		\centering
		\includegraphics{images/Logo}
		\label{fig:Logo}
	\end{minipage}
	\hfill
	\begin{minipage}[t]{4cm}
		\vspace{0pt}
		\scriptsize
		\textbf{DSW \& Partner}\\
		Mr. Mustermann\\
		Projektmanager\\
		Spaßbremen\\
		12345 Spaßstadt\\
		Tel. +00 1234 5678 \\
		E-Mail: bewerbungen@dsw.de
	\end{minipage}
\end{figure}
			
\large
\begin{center}Junior IT Consulant (m/w/divers) \\Business Intelligence\end{center}

\scriptsize
Als Berater im Bereich Business Intelligence erwartet Dich ein internationales Team aus Consultants und Softwareentwicklern aus ganz Europa. In Produktschulungen unserer DSW-Academy und in Weiterbildungsseminaren entwickelst Du Deine technischen und sozialen Kompetenzen weiter. Ob Patenprogramm, flexible Arbeitszeiten oder Home-Office – wir bieten Dir das Arbeitsumfeld, das Du für Deine persönliche Entwicklung benötigst.


\textbf{\\Wir haben spannende Aufgaben für Dich:}
\begin{center}
	\begin{itemize}
		\item Als Consultant für unser Business-Intelligence Softwareprodukt unterstützt Du unsere Kunden remote und vor Ort bei der Einführung unseres Kennzahlen- und ad-hoc-Analyse-Systems im IT-Service Management Kontext
		\item Dein Aufgabenfeld ist breit gefächert. Es reicht von der Analyse und Konzeption über die Umsetzung bis zum Training und der Produkteinführung. Dabei stehen einerseits die technische Beratung sowie andererseits die Umsetzung im Fokus
		\item Du arbeitest in Teilprojekten meist in größeren Implementierungsteams. Damit das Projekt erfolgreich umgesetzt werden kann ist auch der regelmäßige Austausch mit unserem Entwicklungsteam sehr wichtig
	\end{itemize}
\end{center}

\textbf{Das bringst Du mit:}
\begin{center}
	\begin{itemize}
		\item Du bringst ein abgeschlossenes Studium der (Wirtschafts-) Informatik oder vergleichbare Berufsausbildung mit
		\item Du hast bereits erste Berufserfahrung mit Business-Intelligence Systemen oder Dashboard-Systemen
		\item Du bringst Erfahrungen mit relationalen Datenbanken und SQL mit
		\item Du hast Freude daran, zusammen mit unseren Kunden individuelle Lösungen zu entwickeln und diese anschließend zu implementieren
		\item Du hast Spaß an der Zusammenarbeit im Projektteam und bist reisebereit
	\end{itemize}
\end{center}

\begin{center}
	\textbf{Deine Vorteile bei uns:}
\end{center}

\begin{center}
	\includegraphics{images/auto}\label{fig:auto}\hspace{1,5cm}
	\includegraphics{images/home}\label{fig:home}\hspace{1,5cm}	
	\includegraphics{images/handshake}\label{fig:handshake}\hspace{1,5cm}		
	\includegraphics{images/statistics}\label{fig:statistics}  
\end{center}
\hspace{0,3cm}\textbf{Eigener Firmenwagen,\hspace{0,7cm} Flexibel sein durch \hspace{0,9cm}Partnerschaftliches \hspace{1,1cm}Gute Karriere- \& \\ \hspace{0,4cm}auch Privat nutzbar\hspace{1,5cm}Home Office\hspace{2,1cm}Verhältniss \hspace{2cm}Aufstiegschancen} 			

\vspace{1cm}
\textbf{Wir haben dein Interesse geweckt?}\\
Dann schreib deine Bewerbung direkt per Email an mich oder bewirb Dich online über unser Karriere-Portal: www.dombeck-schwarz-warnke.de/DSW/stellenangebote/ – Die Stelle findest Du unter der Kategorie „Berufseinsteiger“.

\normalsize
\newpage
\addcontentsline{toc}{section}{Literaturverzeichniss}
\bibliography{Literaturdatenbank}


\end{flushleft}
\end{document}
