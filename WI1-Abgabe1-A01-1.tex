\documentclass[12pt,utf8]{scrartcl}
\usepackage[ngerman]{babel}
\usepackage{hyperref}
\hypersetup{
	colorlinks=true,	
	linkcolor=blue,     
	citecolor=blue,     
	filecolor=blue,     
	urlcolor=blue     	
}
\usepackage{etoolbox}
\apptocmd{\UrlBreaks}{\do\-\do\%\do\.}{}{}
\usepackage[ngerman]{varioref}
\usepackage{amsmath,amssymb,latexsym,amsfonts,amsthm,amsbsy,qtree}
\usepackage{url}
\usepackage[printonlyused]{acronym}
\usepackage[utf8]{inputenc} 
\usepackage{graphicx}
\usepackage{float}
\usepackage{fancyhdr}
\usepackage{booktabs}
\usepackage{enumitem}
\usepackage[justification=centering]{caption}
\usepackage{natbib}
\bibliographystyle{plainnat}
\usepackage{pdfpages}


\newcommand{\teilnehmerI}{Tom Dombeck}
\newcommand{\mattI}{Matrikelnummer 1} 
\newcommand{\mailI}{todo@uni-bremen.de}
\newcommand{\teilnehmerII}{Andreas Schwarz}
\newcommand{\mattII}{4250572}
\newcommand{\mailII}{andreas4@uni-bremen.de}
\newcommand{\teilnehmerIII}{Lasse Warnke}
\newcommand{\mattIII}{Matrikelnummer 3}
\newcommand{\mailIII}{lwarnke@uni-bremen.de}
\newcommand{\thisgroup}{A01}
\newcommand{\abgabedatum}{19.11.2018}
\newcommand{\nummer}{1}
\newcommand{\thema}{Hype-Cycle-Themen}
\newcommand{\thistutor}{Tim Haß}
\newcommand{\thissemester}{WiSe 2018/19}
\newcommand{\thiscourse}{Wirtschaftsinformatik 1}
\newcommand{\thisshortcourse}{WI1}


\pagestyle{fancy}
\fancyhead{} 												
	\fancyhead[LO,RE]{\thissemester \\ \thisshortcourse} 
	\fancyhead[RO,LE]{TutorIn: \thistutor \\ Gruppe: \thisgroup }
	\fancyfoot{} 											
	\cfoot{\thepage} 										
	\setlength{\headsep}{2cm} 								


\begin{document}
\begin{titlepage}
	\vspace*{\baselineskip}		
	\centering					
	\LARGE							
	\thiscourse \\ 					
	\vspace{1cm}					
	{\Huge 							
	\textbf{Abgabe \nummer: \thema}} \\ 
	\vspace{1.5cm} 					
	TutorIn: \thistutor \\ 		
	\abgabedatum \\ 				
	\vfill 							
	Gruppe: \thisgroup \\ 			
	\vspace{.5cm} 					
	\large 							
	\begin{tabular}{c|c|c} 		
	\teilnehmerI	& \teilnehmerII & \teilnehmerIII \\ 
	\mattI	& \mattII &  \mattIII\\ 
	\mailI	& \mailII & \mailIII \\ 
	\end{tabular} 
\end{titlepage}


\thispagestyle{empty}
\tableofcontents
\newpage
\begin{flushleft}
\setcounter{page}{1}


\section{\label{sec:thema}Aufgabe 1.1: Augmented Reality}
\subsection{\label{sub:thema}a.) Beschreibung von Themegebiet}

...Beschreibung...

\subsection{\label{sub2:thema}b.) eigene Definition}

...Definition...

\subsection{\label{sub3:thema}c.) Ausprägungen und Abwandlungen}

...Beschreibung...

\subsection{\label{sub4:thema}d.) mögliche zukünftige Entwicklung}

...Text...


\section{\label{sec:einfuehrung}Aufgabe 1.2: Unterschiede und Gemeinsamkeiten beim Verständnis von Wirtschaftsinformatik}
\subsection{\label{sub:einfuehrung}a.) Kriterien für einen Vergleich}

...sechs Kriterien....

\subsection{\label{sub2:einfuehrung}b.) Vergleich von den Zeitschriften HMD und BISE}

...Vergleich...

\subsection{\label{sub3:einfuehrung}c.) Vergleich von den Zeitschriften MISQE und ISR}

...Vergleich...

\subsection{\label{sub4:einfuehrung}d.) Vergleich von den Veröffentlichungen aus Deutschland (HMD und BISE) mit denen aus dem internationalen bzw. anglo-amerikanischen Bereich (MISQE und ISR).}

...Vergleich...


\newpage
\section{\label{sec:bonus}Bonusaufgabe}

Die Seite für die Bonusaufgabe....


\newpage
\addcontentsline{toc}{section}{Literaturverzeichniss}
\bibliography{Literaturdatenbank}


\end{flushleft}
\end{document}
